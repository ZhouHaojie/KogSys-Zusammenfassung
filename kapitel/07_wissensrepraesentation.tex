% !TeX root = summary.tex

\section{Wissensrepäsentation}

\subsection{Einführung}

\subsubsection*{Was ist Wissen?}
Gespeicherte
\begin{itemize}
\item Beschreibungen, Modelle, Aktionsfolgen
\item Vorschriften zur Reaktion auf Ereignisse
\item Motorische, kognitive Fähigkeiten
\end{itemize}
konkreter:
\begin{itemize}
\item Computerprogramme
\item Regeln der Aussagenlogik
\item Gewichte von Neuronalen Netzen, \dots
\end{itemize}

\subsection{Grundlagen}

% Bild Folie 8
\mypic{6}{wb1}

Eine Wissensdatenbank\index{Wissensdatenbank} besteht aus Sätzen, die in einer Wissensrepräsentationssprache abgefasst sind. Jeder Satz stellt eine Annahme über die Welt dar:
\begin{center}
\begin{tabular}{ll}
$\alpha$: & "{}\textbf{Ident1} IST-EIN Fahrrad."{} \\ $\beta$: & "{}\textbf{Ident2} IST-EIN Mensch."{} \\ $\gamma$: & "{}\textbf{Ident3} BESITZT Ident1."{} oder \\ $\alpha$: & "{}$r \in R^{20}$."{}
\end{tabular}
\end{center}

% Bild Folie 9
\mypic{8}{wb2}

\begin{description}
\item[Erkläre:] Füge der Datenbank Wissen hinzu.
\item[Befrage:] Frage Wissen aus Datenbank ab.
\end{description}
\begin{center}
\begin{tabular}{|c|}
\hline
Erkläre(DB, "{}Ident1 BESITZT Ident2."{}) \\ $\to$ OK \\ Befrage(DB, "{}Was BESITZT Ident2?"{}) \\ $\to$ Ident1 \\ \hline
\end{tabular}
\end{center}

% Bild Folie 10
\mypic{8}{wb3}

\begin{description}
\item[Deduktion\index{Deduktion}:] Leite neue Sätze von bekannten ab. Injektive Abbildung Deduktion: $$DB \to DB$$ Minimale Deduktion: Identität.
\end{description}

\subsection{Logik allgemein}

\begin{center}
Logik \quad : \quad (Symbolmenge, Modellmenge, Syntax, Semantik, Folgerungsop. $\models$)
\end{center}

\subsubsection*{Symbol-, Modellmenge}

\begin{itemize}
\item Elementarwerte für Aussagen: wahr, falsch
\item Symbolmenge $S$ enthält alle Symbole, über die Aussagen gemacht werden können.
\item Modell: Menge $\{(s_i,w_i)\}$ von Symbolen \\ $s_i \in S$ mit zugeordneten Werten, $w_i \in Def(s_i)$
\item Modellmenge: Menge aller möglichen Modelle: $$\forall i \forall w \in Def(s_i) \exists M \, : \, (s_i,w_i) \in M = (s_i,w)$$
\end{itemize}
\textbf{\textsl{Beispiel: Aussagenlogik}}
\begin{eqnarray*}
S &=& \{a,b\} \\ M &=& \{ \{ (a,wahr),(b,wahr) \} , \\ && \,\,\, \{ (a,wahr),(b,falsch)\} , \\ && \,\,\, \{ (a,falsch),(b,wahr) \} , \\ && \,\,\, \{ (a,falsch) , (b,falsch) \} \}
\end{eqnarray*}
Schreibweise: lateinische Buchstaben für Symbole, griechische Buchstaben für Sätze

\subsubsection*{Syntax}

\begin{description}
\item[Syntax\index{Syntax}:] Legt fest, welche Sätze wohlgeformt sind.
\end{description}
Nur wohlgeformte Sätze sind gültig!
\begin{center}
\begin{tabular}{|r @{$\, \Rightarrow \,$}l @{\quad} l|}
\hline
$(a \vee b)$ & $c$ & wohlgeformt \\
$()a \vee$ & $bc$ & nicht wohlgeformt \\ \hline
\end{tabular}
\end{center}

\subsubsection*{Semantik}

\begin{description}
\item[Semantik\index{Semantik}:] Bestimmt den Wahrheitswert eines Satzes in Bezug auf ein Modell.
\end{description}
\begin{eqnarray*}
\textrm{Satz } \alpha &=& (a \wedge b) \Rightarrow c \\ \textrm{Modell } M &=& \{ (a,wahr) , (b,wahr) , (c,wahr) \} \\ &\Rightarrow& \alpha / M \,\, wahr \, , \, M \textrm{ erfüllt } \alpha \\
\textrm{Modell } M &=& \{ (a,wahr) , (b,wahr) , (c,falsch) \} \\ &\Rightarrow& \alpha / M \,\, falsch
\end{eqnarray*}

\subsubsection*{Folgerung $\models$}

$$\alpha \models \beta \, : \, \textrm{"{}} \beta \textrm{ folgt aus } \alpha \textrm{"{}}$$
$\alpha \models \beta$ genau dann, wenn für alle Modelle, in denen $\alpha$ wahr ist, $\beta$ ebenfalls wahr ist.
$$\alpha \, : \, x+y=4 \qquad \beta_1 \, : \, y = 4-x \qquad \beta_2 \, : \, \frac{x}{y} = 1 \quad \Rightarrow \quad \alpha \models \beta_1 \qquad \alpha \not\models \beta_2$$

Zurück zur Wissensdatenbank $WB$: \\
Kann als $\wedge$-verknüpfte Sequenz von Sätzen aufgefasst werden:
$$WB \, : \, \left\{ \begin{array}{r @{\, : \,} r @{\, = \,} l} \alpha_1 & x+y & 4 \\ \alpha_2 & x/y & 1 \end{array} \right\}$$
$\Rightarrow$ $WB = \alpha_1 \wedge \alpha_2$ $wahr$ $\forall M \, : \, x=2, \, y=2$ \\
$\Rightarrow$ $M$ erfüllt $WB$ $\forall M \, : \, x=2, \, y=2$ \\
$$WB \models \beta \quad \textrm{ gdw } \quad \left( \bigwedge\limits_{\alpha_i \in WB} \alpha_i \right) \Rightarrow \beta = wahr$$

\subsubsection*{Modellprüfung}

Algorithmus zur Überprüfung einer $WB \models \alpha$-Relation:
\begin{enumerate}
\item Finde Menge aller Modelle $M = \{M_i\}$ über $S$
\item Eliminiere alle $M_i$ mit $WB/M = falsch$
\item $WB \models \alpha$ genau dann, wenn $$\forall M_i \in M \, : \, \alpha / M_i = wahr$$
\end{enumerate}

\subsubsection*{Deduktion}

Wenn ein Algorithmus $i$ existiert, der den Satz $\alpha$ aus der Wissensbasis $WB$ ableiten kann, schreiben wir $$WB \vdash_i \alpha$$
("{}$\alpha$ wird durch $i$ aus $WB$ abgeleitet"{} oder "{}$i$ leitet $\alpha$ aus $WB$ ab"{}) \\[0,1cm]
Eigenschaften von Deduktionsalgorithmen:
\begin{itemize}
\item Algorithmus $i$ "{}korrekt"{} genau dann, wenn er \underline{nur} Sätze aus $WB$ ableitet, die aus $WB$ folgen: $$\forall \alpha \, : \, WB \vdash_i \alpha \quad \Rightarrow \quad WB \models \alpha$$
\item Algorithmus $i$ "{}vollständig"{} genau dann, wenn er \underline{alle} Sätze aus $WB$ ableitet, die aus $WB$ folgen: $$\forall \alpha \, : \, WB \models \alpha \quad \Rightarrow \quad WB \vdash_i \alpha$$
\end{itemize}

\subsubsection*{Zusammenfassung}

Eine Logik:
\begin{itemize}
\item bestimmt den Wahrheitsgehalt von Sätzen in Bezug auf Modelle
\item besteht aus Symbolmenge, Modellmenge, Syntax, Semantik, Folgerung
\end{itemize}
Eine Wissensbasis:
\begin{itemize}
\item besteht aus einer Menge von Sätzen
\item passt zu einem Modell $M$ (oder auch nicht)
\end{itemize}
Ein Deduktions-Algorithmus
\begin{itemize}
\item leitet Sätze aus einer Wissensbasis ab
\item kann korrekt und / oder vollständig sein
\end{itemize}

\subsection{Aussagenlogik}

\begin{itemize}
\item Aussagenlogik als Beispiel einer sehr einfachen Logik
\item sehr alt (antikes Griechenland)
\item bildet die Basis, von der die allgemeine Logik abgeleitet wurde
\item eignet sich sehr "{}natürlich"{} zur Illustration bestimmter Konzepte
\end{itemize}

\subsubsection*{Syntax}

\begin{eqnarray*}
Satz &\to& Atom \, | \, Komplex \\ Atom &\to& True \, | \, False \, | \, Symbol \\ Symbol &\to& P \, | \, Q \, | \, R \, \dots \\
Komplex &\to& \neg Satz \, | \, (Satz \wedge Satz) \, | \, (Satz \vee Satz) \, | \, (Satz \Rightarrow Satz) \, | \, (Satz \Leftrightarrow Satz)
\end{eqnarray*}

\subsubsection*{Semantik}
Wahrheitsgehalt im Modell $M$:
\begin{enumerate}
\item $True = wahr$, $False = falsch$ $\forall M$
\item Wahrheitsgehalt von Symbolen wird in $M$ spezifiziert
\item Wahrheitsgehalt aller anderen Sätze rekursiv:
\begin{enumerate}
\item $\forall$ Sätze $\alpha$ $\forall$ Modelle $M$ : $\neg \alpha$ $falsch$ gdw $\alpha \in M$ $wahr$
\item $\forall \alpha,\beta \, \forall M \, : \, \alpha \wedge \beta \,\, wahr$ gdw $\alpha \in M \,\, wahr$ und $\beta \in M \,\, wahr$
\item etc.
\end{enumerate}
(Alternative: Wahrheitstabelle)
\end{enumerate}

\subsubsection*{Muster}

Allgemeingültige Deduktionsregeln aus der klassischen Literatur:
\begin{itemize}
\item \textbf{Modus Ponens\index{Modus Ponens}:} \\ Wenn $\alpha \Rightarrow \beta$ und $\alpha$ gegeben sind, kann $\beta$ inferiert werden. $$\frac{\alpha \Rightarrow \beta \, , \, \alpha}{\beta}$$
\item \textbf{Und-Elimination\index{Und-Elimination}:} $$\frac{\alpha \wedge \beta}{\alpha} \qquad \frac{\alpha \wedge \beta}{\beta}$$ Aus einer Konjunktion kann jedes der Elemente inferiert werden. \\ Korrektheitsbeweis aus Wahrheitstabelle!
\end{itemize}

\subsection{Resolution\index{Resolution}}

Basis für Deduktionsalgorithmen \\
\textbf{Einheits-Resolutionsregel} (aus Modus Ponens): \\ Seien $p_1,\dots,p_k$ und $q$ Literale mit $q = \neg p_i$. Dann gilt: $$\frac{p_1 \vee \dots \vee p_k \, , \, q}{p_1 \vee \dots \vee p_{i-1} \vee p_{i+1} \vee \dots \vee p_k}$$
\textbf{Allgemeine Form} der Resolutionsregel: \\ $p_1,\dots,p_k$ und $q_1,\dots,q_n$ Literale mit $p_i = \neg q_j$. Dann gilt: $$\frac{p_1 \vee \dots \vee p_k \, , \, q_1 \vee \dots \vee q_n}{p_1 \vee \dots \vee p_{i-1} \vee p_{i+1} \vee \dots \vee p_k \vee q_1 \vee \dots \vee q_{j-1} \vee q_{j+1} \vee \dots \vee q_n}$$
Jeder vollständige Suchalgorithmus, kombiniert mit der Resolutionsregel,
\begin{itemize}
\item kann jede Schlussfolgerung ableiten, die aus jeder Wissensbasis der Aussagenlogik folgt.
\item bildet die Basis für Familien von vollständigen Deduktionsalgorithmen.
\item beweist $WB \models \alpha$ durch Widerlegung von $(WB \wedge \neg \alpha)$.
\end{itemize}
Voraussetzung: Sätze müssen in konjunktiver Form vorliegen.
\begin{description}
\item[Klausel\index{Klausel}:] Eine Disjunktion von Literalen $$(l_1 \vee \dots \vee l_n)$$
\item[Konjunktive Form (KF):] Jede aussagenlogische Formel kann als Konjunktion von Klauseln ausgedrückt werden. Dann liegt sie in konjunktiver Form vor: $$(l_{1,1} \vee \dots \vee l_{1,k_1}) \wedge \dots \wedge (l_{n,1} \vee \dots \vee l_{n,k_n})$$
\end{description}

\subsubsection*{Resolutionsalgorithmus}

Zeige $WB \models \alpha$ durch Widerlegung von $(WB \wedge \neg \alpha)$
\begin{center}
\begin{tabular}{l}
$klauseln$ $:=$ Menge Klauseln in KF $(WB \wedge \neg \alpha)$ \\
$neu$ $:=$ $\{\}$ \\
\verb|loop:| \\
\verb|  for all | $C_m$, $C_n$ \verb| in | $klauseln$: \\
\verb|    |$res :=$ \verb| resolution(|$C_m$, $C_n$\verb|)| \\
\verb|    if | $\emptyset$ \verb| = | $res$: \verb| return | $true$ \\
\verb|    |$neu := neu \cup res$ \\
\verb|  if | $neu \subseteq klauseln$: \verb| return | $false$ \\
\verb|  |$klauseln := klauseln \cup neu$
\end{tabular}
\end{center}

Beispiel:
% Bild Folie 34 (Block)
\mypic{10}{bsp_resolution}

Der Resolutionsalgorithmus ist vollständig aber auch sehr afwendig -- $O(n^2)$. Realistische Vereinfachungen:
\begin{itemize}
\item Horn-Klauseln (Prolog), Vorwärts-/Rückwärts-Verkettung
\item Davis-Putnam-Logemann-Loveland (DPLL)
\end{itemize}

\subsection{Effiziente Logik-Algorithmen}

\begin{description}
\item[Horn-Klausel\index{Horn-Klausel}:] Disjunktion von Literalen, von denen höchstens eins positiv ist: $$(\neg a \vee \neg b \vee c)$$ lasst sich schreiben als: $$\underbrace{(a \wedge b)}_{Koerper} \Rightarrow \underbrace{c}_{Kopf}$$
\end{description}
Spezialtypen von Horn-Klauseln:
\begin{itemize}
\item keine negativen Literale: Fakt, Axiom
\item genau ein positives Literal: Definition
\item kein positives Literal: Integritätseinschränkung
\end{itemize}
Eigenschaften von Horn-Klausel-basierten Wissensbasen:
\begin{itemize}
\item einfach visualisierbar: Und-Oder-Graph
\item Inferenz durch Vorwärts- und Rückwärtsverkettung, sehr natürliche und leicht verständliche Algorithmen
\item Folgerungsentscheidung kann in linearer Zeit geschehen!
\end{itemize}
\begin{description}
\item[Und-Oder-Graph\index{Und-Oder-Graph}:] einfache Visualisierung eines Horn-Klausel-Systems.
\end{description}

% Bild Folie 39
\mypic{7}{und_oder_graph}

Vorwärtsverkettung:
\begin{itemize}
\item gehe von Fakten aus
\item arbeite dich vorwärts durch den Baum
\item Ergebnis: alle durchführbaren Folgerungsentscheidungen
\end{itemize}
Rückwärtsverkettung:
\begin{itemize}
\item gehe von Anfrage aus
\item arbeite dich rückwärts durch den Baum
\item stoppe bei bekannten Fakten (Ground Truth, Axiome)
\item Ergebnis: Wahrheitsgehalt der Anfrage
\end{itemize}
\begin{description}
\item[Davis-Putnam-Logemann-Loveland\index{Davis-Putnam-Logemann-Loveland} (DPLL):] Rekursiver Modellprüfungs-Algorithmus mit folgenden Verbesserungen:
\begin{itemize}
\item früher Abbruch: benutzt $(A \vee B) \wedge (A \vee C) = true$, sobald $A = true$
\item Einheitsklauseln (Klauseln mit nur einem Literal): Konjunktionen mit Einheitsklauseln sind nur wahr, wenn die Einheitsklauseln wahr sind \\ $\Rightarrow$ weitere Einschränkung des Suchraums
\item reine Symbole: in $(A \vee \neg B)$, $(\neg B \vee \neg C)$, $(C \vee A)$ sind $A$ und $\neg B$ rein und $C$ unrein \\ Suche nach reinen Symbolen schränkt Raum möglicher Modelle stark ein!
\end{itemize}
\end{description}
Algorithmus:
\begin{center}
\begin{tabular}{l}
\verb|funktion| $DPLL(k,s,m)$: \\
\verb|  # | $k$ Klauselmenge, $s$ Symbolmenge, $m$ Modell \\
\verb|  if| alle Klauseln wahr in $m$: \verb|return | $true$ \\
\verb|  if| eine Klausel falsch in $m$: \verb|return | $false$ \\
\verb|  |$P, \, wert :=$ \verb| FINDE_EINHEITSKLAUSEL(|$k,s,m$\verb|)| \\
\verb|  if | $P$ nicht leer: \\
\verb|    return | $DPLL(k,s-P,$\verb|SETZE_IN_MODELL(|$P,wert,m$\verb|)|$)$ \\
\verb|  |$P, \, wert :=$ \verb| FINDE_REINES_SYMBOL(|$k,m$\verb|)| \\
\verb|  if | $P$ nicht leer: \\
\verb|    return | $DPLL(k,s-P,$\verb|SETZE_IN_MODELL(|$P,wert,m$\verb|)|$)$ \\
\verb|  |$P :=$ \verb| ERSTES(|$s$\verb|); | $rest :=$ \verb| REST(|$s$\verb|)| \\
\verb|  return |$DPLL(k, rest,$\verb|SETZE_IN_MODELL(|$P,true,m$\verb|)|$)$ \verb| or| \\
\verb|         |$DPLL(k, rest,$\verb|SETZE_IN_MODELL(|$P,false,m$\verb|)|$)$
\end{tabular}
\end{center}
DPLL zur Überprüfung von Erfüllbarkeit:
\begin{center}
\begin{tabular}{l}
\verb|function| $DPLL_ERFUELLBAR(s):$ \\
\verb|  #| $s:$ Satz der Aussagenlogik \\
\verb|  | $klauseln :=$ Klauselmenge der KNF von $s$ \\
\verb|  | $symbole :=$ Menge der Symbole aus $s$ \\
\verb|  return| $DPLL(klauseln, symbole, \{\})$
\end{tabular}
\end{center}

\subsection{Semantisches Planen}

\subsubsection*{Repräsentation von Plänen}

Wie kann man Probleme so formulieren, dass
\begin{itemize}
\item ein Lösungsplan einfach zu erstellen ist?
\item die Existenz einer Lösung bewiesen/widerlegt werden kann?
\end{itemize}
$\to$ Einschränkungsregeln für formelle Spezifikation von Zuständen, Zielen, Aktionen. Beispiele solcher Regelsysteme: STRIPS, ADL

\subsubsection*{STRIPS\index{STRIPS}}

\begin{itemize}
\item \textbf{ST}anford \textbf{R}esearch \textbf{I}nstitute \textbf{P}roblem \textbf{S}olver
\item "{}Urvater"{} vieler Planungssysteme
\item sehr einfach aufgebaut
\item teilweise eingeschränkt
\end{itemize}
Repräsentation von Zuständen:
\begin{itemize}
\item Konjunktion positiver aussagenlogischer Literale $$Blau \wedge Rund$$
\item Literale erster Ordnung ($L1$) $$IstTasse(T1) \wedge IstUntertasse(U1) \wedge StehtAuf(T1,U1)$$
\item $L1$ müssen funktions- und variablenfrei sein!
\end{itemize}
\textbf{\textsl{Closed-World Assumption}}
\begin{itemize}
\item geschlossene Welt: was nicht im Zustand vorkommt, ist falsch!
\item keine Negationen im Zustand und Vorbedingungen, aber in Nachbedingungen benötigt:
$$b \quad \Rightarrow \quad \begin{array}{l} a \textrm{ darf nicht im Weltmodell vorkommen} \\ b \textrm{ muss im Weltmodell vorkommen} \end{array}$$
\item Negationen in Nachbedingungen:
$$\neg c \wedge d \quad \Rightarrow \quad \begin{array}{l} \textrm{entferne } c \textrm{ aus dem Weltmodell} \\ \textrm{füge } d \textrm{ dem Weltmodell hinzu} \end{array}$$
\end{itemize}
Repräsentation von Zielen:
\begin{itemize}
\item Ziel: teilweise spezifizierter Zustand
\item angegeben als Konjunktion von positiven Literalen
$$StehtAuf(T1,U1) \wedge StehtAuf(U1,Tisch)$$
\item ein Zustand $S$ erfüllt ein Ziel $Z$, wenn er alle Literale in $Z$ enthält
$$StehtAuf(T1,U1) \wedge StehtAuf(U1,Tisch) \wedge StehtAuf(Teller,Tisch)$$
erfüllt $$StehtAuf(T1,U1) \wedge StehtAuf(U1,Tisch)$$
\end{itemize}
Aktionen werden angegeben durch
\begin{itemize}
\item Aktionsname
\item Parameter
\item Vorbedingungen
\item Effekte
\end{itemize}
$$A = (N_A,P_A,V_A,E_A)$$
\begin{center}
\begin{tabular}{l}
$(StelleAuf,$ \\ $(Obj1,Obj2),$ \\ $IstUntersatz(Obj2) \wedge Auf(Obj1,Tisch) \wedge Auf(Obj2,Tisch),$ \\ $\neg Auf(Obj1,Tisch \wedge Auf(Obj1,Obj2))$
\end{tabular}
\end{center}
Alternative Schreibweise:
\begin{center}
\begin{tabular}{lll}
\textbf{Action} ( & \multicolumn{2}{l}{$StelleAuf(Obj1, Obj2),$} \\
& \textbf{Vorbed:} & $IstUntersatz(Obj2) \wedge$ \\ && $Auf(Obj1, Tisch) \wedge$ \\ && $Auf(Obj2, Tisch),$ \\
& \textbf{Effekt:} & $\neg Auf(Obj1, Tisch) \wedge$ \\ && $Auf(Obj, Obj2)$ \qquad \quad )
\end{tabular}
\end{center}
Semantik:
\begin{description}
\item[Anwendbarkeit:] Eine Aktion ist anwendbar auf allen Modellen $M$, die ein Modell für die Vorbedingung $V_A$ sind.
\item[Ergebnis:] Das Ergebnis $M'$ der Ausführung einer Aktion $A$ auf einem Modell $M$ erhält man durch
\begin{itemize}
\item entfernen aller negativen Literale des Effekts $E_A$ aus $M$
\item hinzufügen aller positiven Literale aus $E_A$ zu $M$.
\end{itemize}
\end{description}
Einschränkunen von STRIPS:
\begin{itemize}
\item Literale müssen funktionsfrei sein
\begin{itemize}
\item endliche Grammatik
\item jede Aktion kann als endliche aussagenlogische Konjunktiondargestellt werden
\item Lösbarkeitsbeweis einfach
\end{itemize}
\item Nur positive Literale $l$, "{}geschlossene Welt"{}
\begin{itemize}
\item einfachere Planung
\item aber: Falsch-Sachverhalte nur implizit
\end{itemize}
\item keine Quantisierung
\end{itemize}
$\Rightarrow$ STRIPS-Aussagen oft lang und unübersichtlich

\subsubsection*{ADL\index{ADL}}

\begin{itemize}
\item \textbf{A}ction \textbf{D}escription \textbf{L}anguage: Weiterentwicklung von STRIPS
\item "{}offene Welt"{}: Was nicht im Zustand vorkommt, ist unbekannt.
\item negative Literale und Disjunktionen erlaubt
\item Effekt $\neg P \wedge Q$:
\begin{itemize}
\item füge $\neg P$ und $Q$ hinzu
\item lösche $P$ und $\neg Q$
\end{itemize}
\item Gleichheits-Prädikat "{}="{} eingebaut
\begin{itemize}
\item $StelleAuf(T1,T1)$ nicht mehr möglich
\end{itemize}
\end{itemize}
Typüberprüfung
\begin{itemize}
\item Typüberprüfung benötigt für Ausführbarkeit bestimmter Aktionen
\item in STRIPS nur explizit (als Prädikat): $IstUntersetzer(x)$, $IstTasse(y)$
\item häufig weggelassen (Schreibarbeit!), aber eigentlich nötig
\item in ADL implizit möglich:
\begin{center}
\begin{tabular}{lll}
\textbf{Action} ( & \multicolumn{2}{l}{$StelleAuf(Obj1 \, : \, Untersatz, \,\, Obj2 \, : \, Manipulierbar),$} \\
& \textbf{Vorbed:} & $Auf(Obj1, Tisch) \wedge$ \\ && $Auf(Obj2, Tisch),$ \\
& \textbf{Effekt:} & $\neg Auf(Obj1, Tisch) \wedge$ \\ && $Auf(Obj, Obj2)$ \qquad \quad )
\end{tabular}
\end{center}
\end{itemize}
Suche im Zustandsraum ist einfach:
\begin{itemize}
\item keine Funktionssymbole \\ $\Rightarrow$ endlicher Zustandsraum \\ $\Rightarrow$ Standard-Algorithmen zur Baumsuche (z.B. A*)
\item Transformation Vorbedingungen $\Leftrightarrow$ Effekte bijektiv \\ $\Rightarrow$ Suche auch rückwärts möglich
\end{itemize}
Aber:
\begin{itemize}
\item Zustandsraum oft sehr groß
\item naive Suche sehr ineffizient
\item benötigt gute Heuristik oder Segmentierung in Teilprobleme
\end{itemize}
Heuristiken für Zustandsraumsuche
\begin{itemize}
\item versuche, die Suchtiefe einzuschränken
\item Def. \textbf{Heuristikfunktion}\index{Heuristikfunktion}: $$h = H(M_0,M_1)$$ angenommene maximale Anzahl von Aktionen, um von $M_0$ zu $M_1$ zu kommen
\item beschränke Suche auf Teilbaum der Tiefe $h$
\item wenn nicht gefunden: suche je eine Ebene tiefer
\item Problem: finde möglichst gute Heuristikfunktionen
\end{itemize}
\subsubsection*{Vorranggraph\index{Vorranggraph}}
Vor- und Nachbedingungen können zur Problemdekomposition verwendet werden. Parralelisierung möglich? $\to$ Vorranggraph \\
Vorranggraph ermöglicht:
\begin{itemize}
\item Parallelisierung der Planung
\item Parallelisierung der Ausführung
\item bei dynamischen Effekten (Hindernisse, etc.)
\begin{itemize}
\item Blockierung des momentanen Teilplans
\item zuerst Ausführung anderer Teilpläne
\item dann Überprüfung, ob blockierter Teilplan jetzt ausführbar (Hindernis jettz weg?)
\item ggf. Neuplanung nur ab der letzten Gabelung notwendig
\end{itemize}
\end{itemize}

\subsection{Umweltmodell\index{Umweltmodell}}

Das Umweltmodell eines kognitiven Systems bildet die reale Umwelt auf eine innere Repräsentation ab. Logisch-semantische Beziehungen genügen für ein System mit Aktorik nicht; zusätzlich:
\begin{itemize}
\item geometrisches Modell: Ausdehung und Lage von Objekten
\item topologisches Modell: (teil-geometrische) Beziehungen zwischen Objekten
\end{itemize}


\begin{description}
\item[Geometrisches Modell:] Anwendungen:
\begin{itemize}
\item Bahnplanung feinkörnig
\item aktives Messen
\item Objekterkennung (Basisobjekte)
\item fahren von $x_s,y_s,z_s$ nach $x_e,y_e,z_e$
\end{itemize}
\item[Topologisches Modell:] Anwendungen:
\begin{itemize}
\item Planung (Manipulationsplanung, mittlere Körnigkeit, Bewegungsplanung)
\item "{}fahre von Raum1 nach Raum5"{}
\end{itemize}
Über topologische Modelle:
\begin{itemize}
\item Anordnung von Objekten und Umwelt relativ zueinander gespeichert
\item abgeleitet aus geometrischen Modellen: \\ Pfade z.B. aus Voronoi.Diagrammen, Quadtree, Potentialfeldern
\item grobe Aktionsplanung aus topologischen Modellen
\item kurzfristige Verwendung anderer Modelle bei unerwarteten Hindernissen (geometrischer Bodenplan, ad-hoc erzeugte Kantenmodelle aus Laserscan, Kamera)
\end{itemize}
\item[Semantisches Modell:] Anwendung: Planung auf Aufgabeneben $\to$ "{}fahre durch alle Büros"{} \\
Über semantische Modelle:
\begin{itemize}
\item besondere Bedeutung im Rahmen der Mensch-Maschine-Kommunikation
\item belegte Fläche als Objekt klassifiziert (z.B Schrank, Stuhl)
\item eingeordnet in topologische Beschreibung
\item Objekteigenschaften verwertbar: Objektzustand (Tür offen, Tasse leer); äußere Erscheinung des Objekts (Form, Farbe) als Diskriminator und Hilfe für die Sensorik; geometrisches Objektmodell für Greifplanung etc.
\end{itemize}
zugehörig zu semantischen Umweltbeschreibungen:
\begin{itemize}
\item Funktion eines Objektes
\item Landmarken
\item geometrische Objekte
\item topologische Umweltdarstellungen
\item Positionen
\end{itemize}
\end{description}

Umweltrepräsentation der Information:
\begin{itemize}
\item Pfade \item Freiraum \item Objekte \item gemischte Modelle
\end{itemize}

\subsubsection*{Pfade}

Normal: 2-dimensionaler Raum
\begin{itemize}
\item Polygonale Beschreibung der Objekte
\item Pfad: lineare oder nichtlineare Verbindung zweier Punkte im Operationsraum
\item Speicherung der Umwelt und Planungsinformation
\item Repräsentation als ungerichteter Graph
\end{itemize}
Sichtbarkeitsgraphen:
\begin{itemize}
\item einfache Umweltdarstellung, partielle Modellierung
\item kollisionsfreie Bewegung nur auf gespeicherten Pfaden
\item Sichtbarkeitsgraphen zur automatischen Generierung von Pfaden
\end{itemize}
Erstellung eines Sichtbarkeitsgraphen:
\begin{center}
\begin{tabular}{l}
\verb|for i = 1..#Obj| \\
\verb|  for j = 1..#Objekte[i]| \\
\verb|    for k = i..#Obj| \\
\verb|      for l = 1..#Objekte[k]| \\
\verb|        if (Sichtbarkeitstest(objekte[i][j], objekte[k],[l] == ok)| \\
\verb|          neuer Pfad(i,j,k,l)|
\end{tabular}
\end{center}
\begin{description}
\item[Voronoi-Diagramme\index{Voronoi-Diagramme}:] Fahrwege liegen auf maximalem Abstand zu den Hindernissen. Das Voronoi-Diagramm ist der duale Graph der Delaunay-Triangulation.
\end{description}

\subsubsection*{Freiraum}

\begin{itemize}
\item Projektion der realen Welt in 2D/3D-Grundrissdarstellung
\item kollisionsfrei befahrbare Freiräume in geeignete Bereiche zerlegt
\item vorhandene Objekte und Hindernisse nicht berücksichtigt
\end{itemize}
Freiraumgraph:
\begin{itemize}
\item Knoten: Freiraumbereiche
\item Kanten: Verbindung der Gebiete
\end{itemize}
Vorteile:
\begin{itemize}
\item geringere Komplexität der Wegplanung
\item reduzierter Zeit- und Rechenaufwand
\item Fahrsicherheitsüberlegungen deutlich vereinfacht
\item Verbesserung der Anpassungsfähigkeit der Algorithmen an verschiedene Umwelten
\end{itemize}
Darstellungsformen der Freiflächen:
\begin{itemize}
\item Kacheln (Quader)
\begin{itemize}
\item Darstellung von besetzten Räumen durch orthogonale begrenzende Linien (Näherungen)
\item Verlängerung der Linien bis zum nächsten besetzten Raum
\item bildet Mosaik von besetzten und freien Kacheln/Quadern
\item freie Kacheln $\to$ Graph, durch den Bewegung möglich ist
\end{itemize}
Quadtree-/Octtree-Aufteilung:
\begin{itemize}
\item hierarchische Aufteilung in freie und belegte Zellen
\item Aufteilung teilbelegter Zellen in 4 (2D) bzw. 8 (3D) Unterzellen
\item Resultat: Baumstruktur
\item Pfadplanung: Aufstieg von Start- und Zielknoten bis zum ersten gemeinsamen Knoten; Liste besuchter Knoten: Pfad
\end{itemize}
\item konvexe Polygone (Polyeder)
\end{itemize}

\subsubsection*{Objekte}

\begin{itemize}
\item Darstellung der Objekte der realen Umwelt (Türen, Wände, Hindernisse)
\item 3-dim. Darstellung der Umgebung aus Sensorwahrnehmungen
\item Projektion auf $x$-$y$-Ebene für Navigation bodengebundener Roboter ausreichend
\end{itemize}
verschiedene Darstellungen:
\begin{itemize}
\item Kantenmodelle: \\
Ermittlung von markanten Punkten. Verbinden durch geeignete Kanten auf Oberfläche des Objekts
\item Oberflächenmodelle
\begin{itemize}
\item Nachbildung der Objektoberflächen
\item Darstellung ebener Flächenelemente mit Polygonen
\item gekrümmte Flächenelemente:
\begin{itemize}
\item mathematische Grundflächen (Zylinder, Kegel, Torusflächen)
\item Bezier-Flächen (Erweiterung des Ansatzes der Bezierkurven): \\
gegeben ist ein Gitter von Führungspunkten $P_{ij}$ mit $0 \leq i \leq N$ und $0 \leq j \leq M$. Damit ist die Fläche beschrieben durch $$F(u,v) = \sum\limits_{i=0}^N \sum\limits_{j=0}^M P_{ij} \cdot B_{i,N}(u) \cdot B_{j,M}(v)$$ mit
\begin{eqnarray*}
B_{i,N}(u) &=& (1-u) B_{i,N-1}(u) + uB_{i+1,N-1}(u) \quad , \quad B_{i,0} = 1 \\
B_{j,M}(v) &=& (1-v) B_{j,M-1}(v) + uB_{j+1,M-1}(v) \quad , \quad B_{j,0} = 1
\end{eqnarray*}
Die $B_{i,N}$ bzw. $B_{j,M}$ heißen auch Bernsteinpolynome.
\item näherungsweise durch Freiformflächen (Patches)
\end{itemize}
\end{itemize}
\item Volumenmodelle:
\begin{itemize}
\item Unterscheidung von Raumpunkten hinsichtlich ihrer Lage zum Objekt (innen-/außenliegend)
\item Repräsentationsmöglichkeiten: Begrenzungsflächenmethode, Grundkörperdarstellung, Zellenzerlegung (Octtree), Volumenapproximation, einhüllende Quader, Geradensegmente
\end{itemize}
\end{itemize}

\subsubsection*{Modelle}

Aufgliederung des benötigten Wissens:
\begin{itemize}
\item Ausführung von Aufgaben
\begin{itemize}
\item semantisches Aufgabenmodell
\item Umweltmodell vorher, nachher
\end{itemize}
\item Bewegung über Grund in 2D, Bewegung des Arms in 3D
\begin{itemize}
\item statische Umweltkarte (geometrisch und topologisch)
\item dynamische Hinderniserfassung in 3D
\item Freiraum-, Hindernismodell
\end{itemize}
\item Manipulation von Objekten
\begin{itemize}
\item Objektpositionen
\item geometrische und topologische Objektmodelle
\end{itemize}
\end{itemize}

\subsection{Geometrisches Planen}

\subsubsection*{Grundlagen der Bahnplanung\index{Bahnplanung}}

Bewegung eines Roboters:
\begin{itemize}
\item Zustandsänderungen über der Zeit (Trajektorie)
\item relativ zu stationärem Koordinatensystem (kartesischer Raum, Gelenkwinkelraum)
\item häufig Gütekriterien, Neben-, Randbedingungen
\end{itemize}
Bekannt:
\begin{itemize}
\item $S_{start}$: Zustand zum Startzeitpunkt
\item $S_{ziel}$: Zustand zum Zielzeitpunkt
\end{itemize}
Gesucht:
\begin{itemize}
\item $S_I$: Zwischenzustände (Stützpunkte)
\item glatte, stetige Trajektorie
\end{itemize}
Bahnplanungsverfahren nach Zustandsraum:
\begin{itemize}
\item Gelenkwinkelzustandsraum (Konfigurationsraum)
\item 3-dim. euklidischer Raum
\item Sensorzustandsraum, Objektzustandsraum, \dots
\end{itemize}
Bahnplanungsverfahren nach Art des Roboters:
\begin{itemize}
\item Bahnplanung für Manipulatoren
\item Bahnplanung für mobile Roboter
\item Bahnplanung für Laufmaschinen und antropomorphe Systeme
\item Greif- und Montageplanung
\end{itemize}

\subsubsection*{Bahnplanung im Gelenkwinkelraum}

\begin{itemize}
\item Trajektorie als Funktion der Gelenkwinkel
\item Ausführung solcher Trajektoren durch
\begin{itemize}
\item Steuerung der Achsen unabhängig voneinander (Punkt zu Punkt) oder
\item achsinterpolierte Steuerung (Bewegung aller Achsen beginnt und endet zum gleichen Zeitpunkt) erfolgen
\end{itemize}
\item Bahnverlauf muss im kartesischen Raum nicht definiert sein
\item Vorteile: einfach; keine Singularitäten
\end{itemize}

\subsubsection*{Bahnplanung im kartesischen Raum}

\begin{itemize}
\item Trajektorie angegeben als Funktion der Endeffektorposition
\item Funktionen z.B.: lineare Bahnen, Polynombahnen, Splines
\item Vorteile:
\begin{itemize}
\item Verlauf der Trajektorie explizit in 3D
\item einfach nachvollziehbar, visualisierbar
\end{itemize}
\item Nachteile:
\begin{itemize}
\item für jeden Punkt muss Gelenkwinkelrücktransformation berechnet werden
\item Trajektorie nicht immer ausführbar (Arbeitsraumbegrenzung, Singularitäten des Roboters)
\end{itemize}
\end{itemize}

\subsubsection*{Bahnplanungs-Schema}

gegeben: Robotermodell (Geometrie, Kinematik), Umweltmodell

\begin{enumerate}
\item Berechnung des Konfigurationsraums $K$
\item Berechnung des Hindernisraums $H$
\item Berechnung des Freiraums $F = K \backslash H$
\item Zerlegung des Freiraums in Unterräume
\item Bahnplanung im Unterraum
\item Integration der lokalen Lösungen in Gasamtlösung
\end{enumerate}

\subsubsection*{Konfiguration}

\begin{description}
\item[Konfiguration $k_R$:] beschreibt den Zustand eines Roboters $R$
\begin{itemize}
\item im euklidischen Raum durch Lage und Orientierung
\item im Gelenkwinkelraum durch die Werte der Gelenke
\end{itemize}
\item[Konfigurationsraum $K_R$:] Raum aller möglichen Konfigurationen von $R$
\item[Weg $w$ von $k_{start}$ bis $k_{ziel}$:] stetige Abbildung $$w \, : \, [0,1] \to K \quad \textrm{mit} \quad w(0) = k_{start} \, , \, w(1) = k_{ziel}$$
\item[Arbeitsraum-Hindernis $h_{A,O}$:] Raum, der von einem Objekt $O$ im Arbeitsraum $A$ eingenommen wird
\item[Konfigurationsraum-Hindernis $h_{K,O}$:] Raum, der von einem Objekt $O$ im Konfigurationsraum $K$ eingenommen wird
\item[Hindernisraum $H$:] \quad $$H = \bigcup\limits_{O} h_{K,O}$$
\end{description}

\subsubsection*{Freiraum}

\begin{itemize}
\item Freiraum $F_R \, : \, F_R = K_R \backslash H$
\item Aufwand für Freiraumberechnung: $O(m^n)$ mit
\begin{itemize}
\item $n$: Anzahl der Freiheitsgrade des Roboters
\item $m$: Anzahl der Hindernisse
\end{itemize}
\item Deshalb oft approximative Verfahren zur Vereinfachung des Freiraums
\begin{itemize}
\item Sichtgraphen
\item Quadtree, Octtree
\end{itemize}
\end{itemize}

\subsection{Bahnplanung in 2D}

Einfacher Algorithmus: "{}Strassenkarten"{} \\
gegeben: 2-dim. Weltmodell, Start und Ziel \\
gesucht: güstigste Verbindung von Start zu Ziel \\
Lösung:
\begin{enumerate}
\item konstruiere Netz von Wegen $W$ in $F_R$
\item bilde $k_{start}$ und $k_{ziel}$ auf $W$ ab: $(W(k_{start}), W(k_{ziel})$
\item suche Weg $w$, der $W(k_{start})$ mit $W(k_{ziel})$ verbindet
\end{enumerate}

\subsubsection*{Wegkonstruktion}

Wegkonstruktion mit
\begin{itemize}
\item Retraktionverfahren (z.B. Voronoi-Diagramm)
\item Sichtgraphen
\item Zellzerlegungsmethoden
\end{itemize}
Suche im Wegnetz mit z.B.
\begin{itemize}
\item A*-Algorithmus (Baumsuche)
\item euklidischer Abstand
\item Potentialfeld
\end{itemize}

\begin{description}
\item[Retraktion\index{Retraktion}:] Sei $X$ eine Menge und $Y \subset X$. Eine surjektive Abbildung $$p \, : \, X \to Y$$ heisst Retraktion genau dann, wenn $p$ stetig ist und $p(y) = y$ für alle $y \in Y$ gilt. \\
D.h. die Abbildung der Menge $X$ auf ihre Teilmenge $Y$, wobei die Menge $Y$ auf sich selbst abgebildet wird. Für die Bahnplanung gilt:
\begin{itemize}
\item $Y$ ist ein Netz von eindimensionalen Kurven (Wegenetz).
\item Retraktionsmethoden unterscheiden sich in Wahl von $p$.
\end{itemize}
\end{description}

\subsubsection*{Sichtgraphen}

Konstruktion:
\begin{itemize}
\item Verbinde jedes Paar von Eckpunkten auf dem Rand von $F_R$ durch gerades Liniensegment, wenn das Segment kein Hindernis schneidet.
\item Verbinde $k_{start}$ mit $k_{ziel}$ analog dazu.
\end{itemize}
Anmerkungen:
\begin{enumerate}
\item Wege sind nur "{}halbfrei"{} (nicht kollisionsfrei), da Hinderniskanten auch Wegsegmente sein können. \\ Abhilfe: Erweiterung der Hindernisse
\item Wenn ein Weg gefunden ist, ist es auch der kürzeste Weg.
\item Methode ist exakt, wenn Roboter nur 2 translatorische Freiheitsgrade hat und sowohl Roboter als auch Hindernisse durch Polygone dargestellt werden können.
\item Methoden auch im $R^3$ anwendbar, jedoch sind die gefundenen Wege i.A. keine kürzesten Wege mehr.
\end{enumerate}

\subsubsection*{Kürzester Pfad im Graphen}

vom Startknoten ausgehend:
\begin{itemize}
\item wähle Nachbarknoten $N_k$ so, dass Evaluationsfunktion $f(N_k)$ minimal
\item suche von $N_k$ ausgehend weiter
\item wenn $f(N_k)$ nicht mehr kleiner wird, mache weiter oben im Baum weiter
\item Problem: $f(N_k)$ darf nicht vom Teilbaum an $N_k$ abhängen! \\ (Würden wir den kennen, bräuchten wir nicht suchen.)
\item Lösung: verwende Heuristik $h(N_k)$
\item beliebt für die Heuristikfunktion in 2D: euklidischer Abstand $$h(N_k) = ||N_k - N_{ziel}||$$
\item rein heuristikbasierte Suche ist nicht optimal und führt nur unter Einschränkungen zum Ziel
\item A*-Algorithmus\index{A*-Algorithmus}: \\ mit $g(N_k)$ Kosten zum Erreichen von $N_k$
\item beweisbar optimal $$f(N_k) = g(N_k) + h(N_k)$$
\end{itemize}













